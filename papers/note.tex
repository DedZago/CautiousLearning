\documentclass{article}
\usepackage{global-layout-notes}
\usepackage{global-statmacros}
\bibliography{~/Documents/MEGA/main/biblio}
\author{Daniele Zago}
\title{Note sul cautious learning a limite fisso con GICP per dati di Poisson}
\date{}
\begin{document}
\maketitle

\section{Poisson GICP}\label{sec:Poisson GICP}
\begin{enumerate}[label=\arabic*)]
    \item  Consideriamo una carta di controllo $ C_{t}$ per aumenti del parametro di una Poisson, 
        \[
            C_{t} = \max\{ 0, \tilde{C}_{t} \},
        \]
        dove $ \tilde{C}_{t}$ è una carta di tipo EWMA (o CUSUM o AEWMA), ad esempio
        \[
            \tilde{C}_{t} = (1 - \lambda)C_{t} + \lambda \frac{x_{t} - \E[X_{t}|\widehat{\theta}_{t - d_{t}}]}{\sqrt{\V[X_{t}|\widehat{\theta}_{t - d_{t}}]}}.
        \]
        Consideriamo il caso del limite di controllo fisso, per cui viene chiamato un allarme se
        \[
          \quad C_{t} \ge L.
        \]
        Per ora consideriamo generica la strategia di update del parametro $ \widehat{\theta}_{t-d_{t}}$ (può essere adaptive estimator, fixed parameter, cautious learning).

    \item Supponiamo che la stima iniziale del parametro sia pari a $ \widehat{\theta}_0$, mentre il vero valore del parametro è $ \theta_0$.
        Allora, la CARL ($ \text{ARL}$ condizionata alla stima $ \widehat{\theta}_0$) della carta di controllo $ C_{t}$ sarà quella di una carta di controllo con vero valore $ \widehat{\theta}_0$ che osserva uno shift fittizio al tempo $ \tau = 1$ di ampiezza $ \theta_0 - \widehat{\theta}_0$.
        In particolare, se la carta unilaterale controlla aumenti del parametro,
        \begin{itemize}
            \item Se $ \widehat{\theta}_0 > \theta_0$, allora lo shift fittizio è $ \theta_0 - \widehat{\theta}_0 < 0$ e quindi $ \text{CARL}_0 > a$.
            \item Se $ \widehat{\theta}_0 < \theta_0$, allora lo shift fittizio è $ \theta_0 - \widehat{\theta}_0 > 0$ e quindi $ \text{CARL}_0 < a$.
        \end{itemize}
\end{enumerate}

Combinando $ 1)$ e $ 2)$ si può disegnare una procedura GICP per la carta di controllo $ C_{t}$, ovvero tale per cui
\begin{equation}
    \label{eq:GICP}
    \Pr(\text{ARL}_0 \le a) = \beta,
\end{equation}
 basata sul seguente procedimento:
\begin{itemize}
    \item Sia $(0, \widehat{\theta}_\text{up})$ un intervallo di confidenza unilaterale di livello $ 1 - \beta$ per $ \theta$.
        Se il parametro stimato è $ \widehat{\theta}$, allora per un fissato $ L = L_0$ la carta di controllo avrà la minima $ \text{CARL}_0$ nell'intervallo di confidenza quando il vero valore del parametro è $\widehat{\theta}_\text{up}$.

    \item Si applichi una procedura (es. \texttt{saControlLimits}) per trovare $ L_\text{up}$, ovvero i limite di controllo che fornisce $ \text{ARL}_0 = a$ per la carta con valore stimato $ \widehat{\theta}$ quando il vero valore del parametro è $ \widehat{\theta}_\text{up}$.

    \item Il limite di controllo $ L_\beta = L_\text{up} $ rende \eqref{eq:GICP} valida poiché l'intervallo di confidenza ha probabilità $ \beta$ di non contenere il vero valore $ \theta_0$.
\end{itemize}

\paragraph{Nota su saControlLimits.} Siccome ci interessa in particolare che valga la \eqref{eq:GICP}, ho fatto una piccola modifica a \texttt{saControlLimits}.
Dall'equazione (12) di \citet{capizzi2016}, usando il parametro di precisione $ \gamma$ si ha che con alta probabilità
\[
    a\cdot (1-\gamma) \le \text{ARL} \le a\cdot (1+\gamma).
\]
Quindi, ottimizzando per $ \text{ARL}_0 = a /(1-\gamma)$ invece di $ \text{ARL}_0 = a$ si ha che con alta probabilità
\begin{equation}
    \label{eq:saCL adjusted}
    a \le \text{ARL} \le a\cdot \frac{1+\gamma}{1 - \gamma},
\end{equation}
il che è più utile se consideriamo la proprietà \eqref{eq:GICP} per la carta.
Per $ a = 500$ e $ \gamma = 0.03$, la \eqref{eq:saCL adjusted} diventa
\[
    500 \le \text{ARL} \le 531.
\]


\section{Cautious Learning}\label{sec:Cautious Learning}
Con la stessa terminologia del paper, si disegna la \textit{cautious region} $ \mathcal{S}_\text{t} = [0, H)$ in modo che
\[
    \text{ATS}_0 = \E[\text{TS}|\tau = \infty ] = s,
\]
per un valore di $ s$ specificato, dove $ \text{TS} = \inf\left\{ n \ge 1 : C_{t}\in \mathcal{S}_{t} \right\}$ è il \textit{time to first stop} dell'update dei parametri.
Per disegnare $ \mathcal{S}_{t}$ basta osservare che $H$ è il limite di controllo di una carta \textit{adaptive estimator} tale per cui $ \text{ARL}_0 = \text{ATS}_0$.
Per cui, si può semplicemente applicare la procedura \texttt{saControlLimits} con \textit{adaptive estimator} per trovare il limite $ H$.
Una volta fissato $ H$, si può applicare la procedura in Sezione~\ref{sec:Poisson GICP} per trovare il limite $ L_\beta$ che fornisce la proprietà GICP \eqref{eq:GICP}.

\paragraph{Osservazione.} Se $ \mathcal{S}_{t} = \left\{ 0 \right\}$, la carta di controllo diventa di tipo \textit{restarting}, ovvero l'update del parametro avviene solo quando $ C_{t} = 0$.
Se poi $ C_{t}$ è una CUSUM, allora siamo nel caso della carta CUSUM-restarting.

% \paragraph{Idea.} (Non testato, pensato giusto) Volendo, si potrebbe disegnare anche la cautious region $ \mathcal{S}_{t}$ con un limite $ H_\beta$ ottenuto usando la procedura GICP, in modo che $ \Pr(\text{ATS}_0 \le s) = \beta.$


\printbibliography
\end{document}

