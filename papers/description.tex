\documentclass{article}
\usepackage{global-layout-notes}
\usepackage{global-statmacros}
\bibliography{~/Documents/MEGA/main/biblio}
\author{Daniele Zago}
\title{Considerazioni}
\date{}
\begin{document}
\maketitle

\section{Poisson GICP}\label{sec:Poisson GICP}
\begin{enumerate}[label=\arabic*)]
    \item  Consideriamo una carta di controllo $ C_{t}$ di tipo EWMA (o CUSUM o AEWMA) a limite fisso che suona un allarme per
        \[
          C_{t} \le -L \quad \text{oppure} \quad C_{t} \ge L.
        \]

    \item Supponiamo che la stima del parametro sia pari a $ \widehat{\theta}$, mentre il vero parametro ha valore $ \theta_0$.
        Allora, la ARL della carta di controllo $ C_{t}$ sarà quella di una carta di controllo con vero valore $ \widehat{\theta}$ che osserva uno shift al tempo $ \tau = 1$ di ampiezza $ \widehat{\theta} - \theta_0$.
\end{enumerate}

Combinando $ 1)$ e $ 2)$ si può disegnare una procedura GICP, ovvero tale per cui
\begin{equation}
    \label{eq:GICP}
    \Pr(\text{ARL}_0 \le a) = \beta
\end{equation}
per la carta di controllo $ C_{t}$ basata sul seguente procedimento:
\begin{itemize}
    \item Sia $(\widehat{\theta}_\text{low}, \widehat{\theta}_\text{up})$ un intervallo di confidenza di livello $ 1 - \beta$ per $ \theta$.
        Se il parametro stimato è $ \widehat{\theta}$, per un fissato $ L = L_0$ la carta di controllo avrà la minima $ \text{ARL}_0$ nell'intervallo di confidenza se il vero valore del parametro è $ \widehat{\theta}_\text{low}$ oppure $ \widehat{\theta}_\text{up}$.

    \item Si applichi una procedura (es. \texttt{saControlLimits}) per trovare $ L_\text{low}$ e $ L_\text{up}$, ovvero i limiti di controllo che forniscono $ \text{ARL}_0 = a$ per la carta con valore stimato $ \widehat{\theta}$ quando il vero valore del parametro è, rispettivamente, $ \widehat{\theta}_\text{low}$ e $ \widehat{\theta}_\text{up}$.

    \item Il limite di controllo più conservativo, $ L_{\beta} = \max\left\{ L_\text{low}, L_\text{up} \right\}$, rende \eqref{eq:GICP} valida poiché l'intervallo di confidenza ha probabilità $ \beta$ di non contenere il vero valore $ \theta_0$.
\end{itemize}

\paragraph{Nota su saControlLimits.} Siccome ci interessa in particolare che valga la \eqref{eq:GICP}, ho fatto una piccola modifica a \texttt{saControlLimits}.
Dall'equazione (12) di \citep{capizzi2016}, usando il parametro di precisione $ \gamma$ si ha che con alta probabilità
\[
    a\cdot (1-\gamma) \le \text{ARL} \le a\cdot (1+\gamma).
\]
Quindi, ottimizzando per $ \text{ARL}_0 = a /(1-\gamma)$ invece di $ \text{ARL}_0 = a$ si ha che con alta probabilità
\[
    a \le \text{ARL} \le a\cdot (1-\gamma)(1+\gamma),
\]
il che è più in accordo con la proprietà \eqref{eq:GICP}.

\section{Cautious Learning}\label{sec:Cautious Learning}
Con la stessa terminologia del paper, si disegna la \textit{cautious region} $ \mathcal{S}_\text{t} = (-H, H)$ in modo che
\[
    \text{ATS}_0 = \E[TS|\tau = \infty ] = s,
\]
per un valore di $ s$ specificato, dove $ \text{TS} = \inf\left\{ n \ge 1 : C_{t}\in \mathcal{S}_{t} \right\}$ è il \textit{time to first stop} dell'update dei parametri.
Per disegnare $ \mathcal{S}_{t}$ basta osservare che $H$ è il limite di controllo di una carta \textit{adaptive estimator} tale per cui $ \text{ARL}_0 = \text{ATS}_0$.
Per cui, si può semplicemente applicare la procedura \texttt{saControlLimits} con \textit{adaptive estimator} per trovare il limite $ H$.
Una volta fissato $ H$, si può applicare la procedura in Sezione~\ref{sec:Poisson GICP} per trovare il limite $ L_\beta$ che fornisce la proprietà GICP \eqref{eq:GICP}.

\paragraph{Idea.} Volendo, si potrebbe disegnare anche la cautious region $ \mathcal{S}_{t}$ usando la procedura GICP, in modo che $ \Pr(\text{ATS}_0 \le s) = \beta.$


\printbibliography
\end{document}

